% Generated by Sphinx.
\def\sphinxdocclass{report}
\newif\ifsphinxKeepOldNames \sphinxKeepOldNamestrue
\documentclass[letterpaper,10pt,english]{sphinxmanual}
\usepackage{iftex}

\ifPDFTeX
  \usepackage[utf8]{inputenc}
\fi
\ifdefined\DeclareUnicodeCharacter
  \DeclareUnicodeCharacter{00A0}{\nobreakspace}
\fi
\usepackage{cmap}
\usepackage[T1]{fontenc}
\usepackage{amsmath,amssymb,amstext}
\usepackage{babel}
\usepackage{times}
\usepackage[Bjarne]{fncychap}
\usepackage{longtable}
\usepackage{sphinx}
\usepackage{multirow}
\usepackage{eqparbox}

\addto\captionsenglish{\renewcommand{\contentsname}{Contents:}}

\addto\captionsenglish{\renewcommand{\figurename}{Fig.\@ }}
\addto\captionsenglish{\renewcommand{\tablename}{Table }}
\SetupFloatingEnvironment{literal-block}{name=Listing }

\addto\extrasenglish{\def\pageautorefname{page}}

\setcounter{tocdepth}{1}


\title{Python\_GedLib Documentation}
\date{Jul 22, 2019}
\release{1.0}
\author{Natacha Lambert}
\newcommand{\sphinxlogo}{}
\renewcommand{\releasename}{Release}
\makeindex

\makeatletter
\def\PYG@reset{\let\PYG@it=\relax \let\PYG@bf=\relax%
    \let\PYG@ul=\relax \let\PYG@tc=\relax%
    \let\PYG@bc=\relax \let\PYG@ff=\relax}
\def\PYG@tok#1{\csname PYG@tok@#1\endcsname}
\def\PYG@toks#1+{\ifx\relax#1\empty\else%
    \PYG@tok{#1}\expandafter\PYG@toks\fi}
\def\PYG@do#1{\PYG@bc{\PYG@tc{\PYG@ul{%
    \PYG@it{\PYG@bf{\PYG@ff{#1}}}}}}}
\def\PYG#1#2{\PYG@reset\PYG@toks#1+\relax+\PYG@do{#2}}

\expandafter\def\csname PYG@tok@w\endcsname{\def\PYG@tc##1{\textcolor[rgb]{0.73,0.73,0.73}{##1}}}
\expandafter\def\csname PYG@tok@c\endcsname{\let\PYG@it=\textit\def\PYG@tc##1{\textcolor[rgb]{0.25,0.50,0.56}{##1}}}
\expandafter\def\csname PYG@tok@cp\endcsname{\def\PYG@tc##1{\textcolor[rgb]{0.00,0.44,0.13}{##1}}}
\expandafter\def\csname PYG@tok@cs\endcsname{\def\PYG@tc##1{\textcolor[rgb]{0.25,0.50,0.56}{##1}}\def\PYG@bc##1{\setlength{\fboxsep}{0pt}\colorbox[rgb]{1.00,0.94,0.94}{\strut ##1}}}
\expandafter\def\csname PYG@tok@k\endcsname{\let\PYG@bf=\textbf\def\PYG@tc##1{\textcolor[rgb]{0.00,0.44,0.13}{##1}}}
\expandafter\def\csname PYG@tok@kp\endcsname{\def\PYG@tc##1{\textcolor[rgb]{0.00,0.44,0.13}{##1}}}
\expandafter\def\csname PYG@tok@kt\endcsname{\def\PYG@tc##1{\textcolor[rgb]{0.56,0.13,0.00}{##1}}}
\expandafter\def\csname PYG@tok@o\endcsname{\def\PYG@tc##1{\textcolor[rgb]{0.40,0.40,0.40}{##1}}}
\expandafter\def\csname PYG@tok@ow\endcsname{\let\PYG@bf=\textbf\def\PYG@tc##1{\textcolor[rgb]{0.00,0.44,0.13}{##1}}}
\expandafter\def\csname PYG@tok@nb\endcsname{\def\PYG@tc##1{\textcolor[rgb]{0.00,0.44,0.13}{##1}}}
\expandafter\def\csname PYG@tok@nf\endcsname{\def\PYG@tc##1{\textcolor[rgb]{0.02,0.16,0.49}{##1}}}
\expandafter\def\csname PYG@tok@nc\endcsname{\let\PYG@bf=\textbf\def\PYG@tc##1{\textcolor[rgb]{0.05,0.52,0.71}{##1}}}
\expandafter\def\csname PYG@tok@nn\endcsname{\let\PYG@bf=\textbf\def\PYG@tc##1{\textcolor[rgb]{0.05,0.52,0.71}{##1}}}
\expandafter\def\csname PYG@tok@ne\endcsname{\def\PYG@tc##1{\textcolor[rgb]{0.00,0.44,0.13}{##1}}}
\expandafter\def\csname PYG@tok@nv\endcsname{\def\PYG@tc##1{\textcolor[rgb]{0.73,0.38,0.84}{##1}}}
\expandafter\def\csname PYG@tok@no\endcsname{\def\PYG@tc##1{\textcolor[rgb]{0.38,0.68,0.84}{##1}}}
\expandafter\def\csname PYG@tok@nl\endcsname{\let\PYG@bf=\textbf\def\PYG@tc##1{\textcolor[rgb]{0.00,0.13,0.44}{##1}}}
\expandafter\def\csname PYG@tok@ni\endcsname{\let\PYG@bf=\textbf\def\PYG@tc##1{\textcolor[rgb]{0.84,0.33,0.22}{##1}}}
\expandafter\def\csname PYG@tok@na\endcsname{\def\PYG@tc##1{\textcolor[rgb]{0.25,0.44,0.63}{##1}}}
\expandafter\def\csname PYG@tok@nt\endcsname{\let\PYG@bf=\textbf\def\PYG@tc##1{\textcolor[rgb]{0.02,0.16,0.45}{##1}}}
\expandafter\def\csname PYG@tok@nd\endcsname{\let\PYG@bf=\textbf\def\PYG@tc##1{\textcolor[rgb]{0.33,0.33,0.33}{##1}}}
\expandafter\def\csname PYG@tok@s\endcsname{\def\PYG@tc##1{\textcolor[rgb]{0.25,0.44,0.63}{##1}}}
\expandafter\def\csname PYG@tok@sd\endcsname{\let\PYG@it=\textit\def\PYG@tc##1{\textcolor[rgb]{0.25,0.44,0.63}{##1}}}
\expandafter\def\csname PYG@tok@si\endcsname{\let\PYG@it=\textit\def\PYG@tc##1{\textcolor[rgb]{0.44,0.63,0.82}{##1}}}
\expandafter\def\csname PYG@tok@se\endcsname{\let\PYG@bf=\textbf\def\PYG@tc##1{\textcolor[rgb]{0.25,0.44,0.63}{##1}}}
\expandafter\def\csname PYG@tok@sr\endcsname{\def\PYG@tc##1{\textcolor[rgb]{0.14,0.33,0.53}{##1}}}
\expandafter\def\csname PYG@tok@ss\endcsname{\def\PYG@tc##1{\textcolor[rgb]{0.32,0.47,0.09}{##1}}}
\expandafter\def\csname PYG@tok@sx\endcsname{\def\PYG@tc##1{\textcolor[rgb]{0.78,0.36,0.04}{##1}}}
\expandafter\def\csname PYG@tok@m\endcsname{\def\PYG@tc##1{\textcolor[rgb]{0.13,0.50,0.31}{##1}}}
\expandafter\def\csname PYG@tok@gh\endcsname{\let\PYG@bf=\textbf\def\PYG@tc##1{\textcolor[rgb]{0.00,0.00,0.50}{##1}}}
\expandafter\def\csname PYG@tok@gu\endcsname{\let\PYG@bf=\textbf\def\PYG@tc##1{\textcolor[rgb]{0.50,0.00,0.50}{##1}}}
\expandafter\def\csname PYG@tok@gd\endcsname{\def\PYG@tc##1{\textcolor[rgb]{0.63,0.00,0.00}{##1}}}
\expandafter\def\csname PYG@tok@gi\endcsname{\def\PYG@tc##1{\textcolor[rgb]{0.00,0.63,0.00}{##1}}}
\expandafter\def\csname PYG@tok@gr\endcsname{\def\PYG@tc##1{\textcolor[rgb]{1.00,0.00,0.00}{##1}}}
\expandafter\def\csname PYG@tok@ge\endcsname{\let\PYG@it=\textit}
\expandafter\def\csname PYG@tok@gs\endcsname{\let\PYG@bf=\textbf}
\expandafter\def\csname PYG@tok@gp\endcsname{\let\PYG@bf=\textbf\def\PYG@tc##1{\textcolor[rgb]{0.78,0.36,0.04}{##1}}}
\expandafter\def\csname PYG@tok@go\endcsname{\def\PYG@tc##1{\textcolor[rgb]{0.20,0.20,0.20}{##1}}}
\expandafter\def\csname PYG@tok@gt\endcsname{\def\PYG@tc##1{\textcolor[rgb]{0.00,0.27,0.87}{##1}}}
\expandafter\def\csname PYG@tok@err\endcsname{\def\PYG@bc##1{\setlength{\fboxsep}{0pt}\fcolorbox[rgb]{1.00,0.00,0.00}{1,1,1}{\strut ##1}}}
\expandafter\def\csname PYG@tok@kc\endcsname{\let\PYG@bf=\textbf\def\PYG@tc##1{\textcolor[rgb]{0.00,0.44,0.13}{##1}}}
\expandafter\def\csname PYG@tok@kd\endcsname{\let\PYG@bf=\textbf\def\PYG@tc##1{\textcolor[rgb]{0.00,0.44,0.13}{##1}}}
\expandafter\def\csname PYG@tok@kn\endcsname{\let\PYG@bf=\textbf\def\PYG@tc##1{\textcolor[rgb]{0.00,0.44,0.13}{##1}}}
\expandafter\def\csname PYG@tok@kr\endcsname{\let\PYG@bf=\textbf\def\PYG@tc##1{\textcolor[rgb]{0.00,0.44,0.13}{##1}}}
\expandafter\def\csname PYG@tok@bp\endcsname{\def\PYG@tc##1{\textcolor[rgb]{0.00,0.44,0.13}{##1}}}
\expandafter\def\csname PYG@tok@vc\endcsname{\def\PYG@tc##1{\textcolor[rgb]{0.73,0.38,0.84}{##1}}}
\expandafter\def\csname PYG@tok@vg\endcsname{\def\PYG@tc##1{\textcolor[rgb]{0.73,0.38,0.84}{##1}}}
\expandafter\def\csname PYG@tok@vi\endcsname{\def\PYG@tc##1{\textcolor[rgb]{0.73,0.38,0.84}{##1}}}
\expandafter\def\csname PYG@tok@sb\endcsname{\def\PYG@tc##1{\textcolor[rgb]{0.25,0.44,0.63}{##1}}}
\expandafter\def\csname PYG@tok@sc\endcsname{\def\PYG@tc##1{\textcolor[rgb]{0.25,0.44,0.63}{##1}}}
\expandafter\def\csname PYG@tok@s2\endcsname{\def\PYG@tc##1{\textcolor[rgb]{0.25,0.44,0.63}{##1}}}
\expandafter\def\csname PYG@tok@sh\endcsname{\def\PYG@tc##1{\textcolor[rgb]{0.25,0.44,0.63}{##1}}}
\expandafter\def\csname PYG@tok@s1\endcsname{\def\PYG@tc##1{\textcolor[rgb]{0.25,0.44,0.63}{##1}}}
\expandafter\def\csname PYG@tok@mb\endcsname{\def\PYG@tc##1{\textcolor[rgb]{0.13,0.50,0.31}{##1}}}
\expandafter\def\csname PYG@tok@mf\endcsname{\def\PYG@tc##1{\textcolor[rgb]{0.13,0.50,0.31}{##1}}}
\expandafter\def\csname PYG@tok@mh\endcsname{\def\PYG@tc##1{\textcolor[rgb]{0.13,0.50,0.31}{##1}}}
\expandafter\def\csname PYG@tok@mi\endcsname{\def\PYG@tc##1{\textcolor[rgb]{0.13,0.50,0.31}{##1}}}
\expandafter\def\csname PYG@tok@il\endcsname{\def\PYG@tc##1{\textcolor[rgb]{0.13,0.50,0.31}{##1}}}
\expandafter\def\csname PYG@tok@mo\endcsname{\def\PYG@tc##1{\textcolor[rgb]{0.13,0.50,0.31}{##1}}}
\expandafter\def\csname PYG@tok@ch\endcsname{\let\PYG@it=\textit\def\PYG@tc##1{\textcolor[rgb]{0.25,0.50,0.56}{##1}}}
\expandafter\def\csname PYG@tok@cm\endcsname{\let\PYG@it=\textit\def\PYG@tc##1{\textcolor[rgb]{0.25,0.50,0.56}{##1}}}
\expandafter\def\csname PYG@tok@cpf\endcsname{\let\PYG@it=\textit\def\PYG@tc##1{\textcolor[rgb]{0.25,0.50,0.56}{##1}}}
\expandafter\def\csname PYG@tok@c1\endcsname{\let\PYG@it=\textit\def\PYG@tc##1{\textcolor[rgb]{0.25,0.50,0.56}{##1}}}

\def\PYGZbs{\char`\\}
\def\PYGZus{\char`\_}
\def\PYGZob{\char`\{}
\def\PYGZcb{\char`\}}
\def\PYGZca{\char`\^}
\def\PYGZam{\char`\&}
\def\PYGZlt{\char`\<}
\def\PYGZgt{\char`\>}
\def\PYGZsh{\char`\#}
\def\PYGZpc{\char`\%}
\def\PYGZdl{\char`\$}
\def\PYGZhy{\char`\-}
\def\PYGZsq{\char`\'}
\def\PYGZdq{\char`\"}
\def\PYGZti{\char`\~}
% for compatibility with earlier versions
\def\PYGZat{@}
\def\PYGZlb{[}
\def\PYGZrb{]}
\makeatother

\renewcommand\PYGZsq{\textquotesingle}

\begin{document}

\maketitle
\tableofcontents
\phantomsection\label{index::doc}

\phantomsection\label{doc:module-PythonGedLib}\index{PythonGedLib (module)}

\chapter{Python GedLib module}
\label{doc:welcome-to-python-gedlib-s-documentation}\label{doc::doc}\label{doc:python-gedlib-module}
This module allow to use a C++ library for edit distance between graphs (GedLib) with Python.


\section{Authors}
\label{doc:authors}
David Blumenthal
Natacha Lambert

Copyright (C) 2019 by all the authors


\section{Classes \& Functions}
\label{doc:classes-functions}\index{EditCostError}

\begin{fulllineitems}
\phantomsection\label{doc:PythonGedLib.EditCostError}\pysigline{\sphinxstrong{exception }\sphinxcode{PythonGedLib.}\sphinxbfcode{EditCostError}}
Class for Edit Cost Error. Raise an error if an edit cost function doesn't exist in the library (not in listOfEditCostOptions).
\begin{quote}\begin{description}
\item[{Attribute message}] \leavevmode
The message to print when an error is detected.

\end{description}\end{quote}

\end{fulllineitems}

\index{Error}

\begin{fulllineitems}
\phantomsection\label{doc:PythonGedLib.Error}\pysigline{\sphinxstrong{exception }\sphinxcode{PythonGedLib.}\sphinxbfcode{Error}}
Class for error's management. This one is general.

\end{fulllineitems}

\index{InitError}

\begin{fulllineitems}
\phantomsection\label{doc:PythonGedLib.InitError}\pysigline{\sphinxstrong{exception }\sphinxcode{PythonGedLib.}\sphinxbfcode{InitError}}
Class for Init Error. Raise an error if an init option doesn't exist in the library (not in listOfInitOptions).
\begin{quote}\begin{description}
\item[{Attribute message}] \leavevmode
The message to print when an error is detected.

\end{description}\end{quote}

\end{fulllineitems}

\index{MethodError}

\begin{fulllineitems}
\phantomsection\label{doc:PythonGedLib.MethodError}\pysigline{\sphinxstrong{exception }\sphinxcode{PythonGedLib.}\sphinxbfcode{MethodError}}
Class for Method Error. Raise an error if a computation method doesn't exist in the library (not in listOfMethodOptions).
\begin{quote}\begin{description}
\item[{Attribute message}] \leavevmode
The message to print when an error is detected.

\end{description}\end{quote}

\end{fulllineitems}

\index{PyAddEdge() (in module PythonGedLib)}

\begin{fulllineitems}
\phantomsection\label{doc:PythonGedLib.PyAddEdge}\pysiglinewithargsret{\sphinxcode{PythonGedLib.}\sphinxbfcode{PyAddEdge}}{}{}
Adds an edge on a graph selected by its ID.
\begin{quote}\begin{description}
\item[{Parameters}] \leavevmode\begin{itemize}
\item {} 
\textbf{\texttt{graphID}} (\emph{\texttt{size\_t}}) -- The ID of the wanted graph

\item {} 
\textbf{\texttt{tail}} (\href{https://docs.python.org/3/library/string.html\#module-string}{\emph{\texttt{string}}}) -- The ID of the tail node for the new edge

\item {} 
\textbf{\texttt{head}} (\href{https://docs.python.org/3/library/string.html\#module-string}{\emph{\texttt{string}}}) -- The ID of the head node for the new edge

\item {} 
\textbf{\texttt{edgeLabel}} (\emph{\texttt{dict\{string : string\}}}) -- The label of the new edge

\item {} 
\textbf{\texttt{ignoreDuplicates}} (\href{https://docs.python.org/3/library/functions.html\#bool}{\emph{\texttt{bool}}}) -- If True, duplicate edges are ignored, otherwise it's raise an error if an existing edge is added. True by default

\end{itemize}

\end{description}\end{quote}

\begin{notice}{note}{Note:}
You can also use this function after initialization, but only on a newly added graph. Call PyInitEnv() after you're finished your modifications.
\end{notice}

\end{fulllineitems}

\index{PyAddGraph() (in module PythonGedLib)}

\begin{fulllineitems}
\phantomsection\label{doc:PythonGedLib.PyAddGraph}\pysiglinewithargsret{\sphinxcode{PythonGedLib.}\sphinxbfcode{PyAddGraph}}{}{}
Adds a empty graph on the environment, with its name and its class. Nodes and edges will be add in a second time.
\begin{quote}\begin{description}
\item[{Parameters}] \leavevmode\begin{itemize}
\item {} 
\textbf{\texttt{name}} (\href{https://docs.python.org/3/library/string.html\#module-string}{\emph{\texttt{string}}}) -- The name of the new graph, an empty string by default

\item {} 
\textbf{\texttt{classe}} (\href{https://docs.python.org/3/library/string.html\#module-string}{\emph{\texttt{string}}}) -- The class of the new graph, an empty string by default

\end{itemize}

\item[{Returns}] \leavevmode
The ID of the newly graphe

\item[{Return type}] \leavevmode
size\_t

\end{description}\end{quote}

\begin{notice}{note}{Note:}
You can call this function without parameters. You can also use this function after initialization, call PyInitEnv() after you're finished your modifications.
\end{notice}

\end{fulllineitems}

\index{PyAddNode() (in module PythonGedLib)}

\begin{fulllineitems}
\phantomsection\label{doc:PythonGedLib.PyAddNode}\pysiglinewithargsret{\sphinxcode{PythonGedLib.}\sphinxbfcode{PyAddNode}}{}{}
Adds a node on a graph selected by its ID. A ID and a label for the node is required.
\begin{quote}\begin{description}
\item[{Parameters}] \leavevmode\begin{itemize}
\item {} 
\textbf{\texttt{graphID}} (\emph{\texttt{size\_t}}) -- The ID of the wanted graph

\item {} 
\textbf{\texttt{nodeID}} (\href{https://docs.python.org/3/library/string.html\#module-string}{\emph{\texttt{string}}}) -- The ID of the new node

\item {} 
\textbf{\texttt{nodeLabel}} (\emph{\texttt{dict\{string : string\}}}) -- The label of the new node

\end{itemize}

\end{description}\end{quote}

\begin{notice}{note}{Note:}
You can also use this function after initialization, but only on a newly added graph. Call PyInitEnv() after you're finished your modifications.
\end{notice}

\end{fulllineitems}

\index{PyAddSymmetricalEdge() (in module PythonGedLib)}

\begin{fulllineitems}
\phantomsection\label{doc:PythonGedLib.PyAddSymmetricalEdge}\pysiglinewithargsret{\sphinxcode{PythonGedLib.}\sphinxbfcode{PyAddSymmetricalEdge}}{}{}
Adds a symmetrical edge on a graph selected by its ID.
\begin{quote}\begin{description}
\item[{Parameters}] \leavevmode\begin{itemize}
\item {} 
\textbf{\texttt{graphID}} (\emph{\texttt{size\_t}}) -- The ID of the wanted graph

\item {} 
\textbf{\texttt{tail}} (\href{https://docs.python.org/3/library/string.html\#module-string}{\emph{\texttt{string}}}) -- The ID of the tail node for the new edge

\item {} 
\textbf{\texttt{head}} (\href{https://docs.python.org/3/library/string.html\#module-string}{\emph{\texttt{string}}}) -- The ID of the head node for the new edge

\item {} 
\textbf{\texttt{edgeLabel}} (\emph{\texttt{dict\{string : string\}}}) -- The label of the new edge

\end{itemize}

\end{description}\end{quote}

\begin{notice}{note}{Note:}
You can also use this function after initialization, but only on a newly added graph. Call PyInitEnv() after you're finished your modifications.
\end{notice}

\end{fulllineitems}

\index{PyClearGraph() (in module PythonGedLib)}

\begin{fulllineitems}
\phantomsection\label{doc:PythonGedLib.PyClearGraph}\pysiglinewithargsret{\sphinxcode{PythonGedLib.}\sphinxbfcode{PyClearGraph}}{}{}
Deletes a graph, selected by its ID, to the environment.
\begin{quote}\begin{description}
\item[{Parameters}] \leavevmode
\textbf{\texttt{graphID}} (\emph{\texttt{size\_t}}) -- The ID of the wanted graph

\end{description}\end{quote}

\begin{notice}{note}{Note:}
Call PyInit() after you're finished your modifications.
\end{notice}

\end{fulllineitems}

\index{PyGetAllGraphIds() (in module PythonGedLib)}

\begin{fulllineitems}
\phantomsection\label{doc:PythonGedLib.PyGetAllGraphIds}\pysiglinewithargsret{\sphinxcode{PythonGedLib.}\sphinxbfcode{PyGetAllGraphIds}}{}{}
Searchs all the IDs of the loaded graphs in the environment.
\begin{quote}\begin{description}
\item[{Returns}] \leavevmode
The list of all graphs's Ids

\item[{Return type}] \leavevmode
list{[}size\_t{]}

\end{description}\end{quote}

\begin{notice}{note}{Note:}
The last ID is equal to (number of graphs - 1). The order correspond to the loading order.
\end{notice}

\end{fulllineitems}

\index{PyGetAllMap() (in module PythonGedLib)}

\begin{fulllineitems}
\phantomsection\label{doc:PythonGedLib.PyGetAllMap}\pysiglinewithargsret{\sphinxcode{PythonGedLib.}\sphinxbfcode{PyGetAllMap}}{}{}~\begin{quote}

Returns a vector which contains the forward and the backward maps between nodes of the two indicated graphs.
\end{quote}
\begin{quote}\begin{description}
\item[{Parameters}] \leavevmode\begin{itemize}
\item {} 
\textbf{\texttt{g}} (\emph{\texttt{size\_t}}) -- The Id of the first compared graph

\item {} 
\textbf{\texttt{h}} (\emph{\texttt{size\_t}}) -- The Id of the second compared graph

\end{itemize}

\item[{Returns}] \leavevmode
The forward and backward maps to the adjacence matrix between nodes of the two graphs

\item[{Return type}] \leavevmode
list{[}list{[}npy\_uint32{]}{]}

\end{description}\end{quote}


\sphinxstrong{See also:}


PyRunMethod(), PyGetUpperBound(), PyGetLowerBound(),  PyGetForwardMap(), PyGetBackwardMap(), PyGetRuntime(), PyQuasimetricCost()



\begin{notice}{warning}{Warning:}
PyRunMethod() between the same two graph must be called before this function.
\end{notice}

\begin{notice}{note}{Note:}
This function duplicates data so please don't use it. I also don't know how to connect the two map to reconstruct the adjacence matrix. Please come back when I know how it's work !
\end{notice}

\end{fulllineitems}

\index{PyGetAssignmentMatrix() (in module PythonGedLib)}

\begin{fulllineitems}
\phantomsection\label{doc:PythonGedLib.PyGetAssignmentMatrix}\pysiglinewithargsret{\sphinxcode{PythonGedLib.}\sphinxbfcode{PyGetAssignmentMatrix}}{}{}
Returns the Assignment Matrix between two selected graphs g and h.
\begin{quote}\begin{description}
\item[{Parameters}] \leavevmode\begin{itemize}
\item {} 
\textbf{\texttt{g}} (\emph{\texttt{size\_t}}) -- The Id of the first compared graph

\item {} 
\textbf{\texttt{h}} (\emph{\texttt{size\_t}}) -- The Id of the second compared graph

\end{itemize}

\item[{Returns}] \leavevmode
The Assignment Matrix between the two selected graph.

\item[{Return type}] \leavevmode
list{[}list{[}int{]}{]}

\end{description}\end{quote}


\sphinxstrong{See also:}


PyRunMethod(), PyGetForwardMap(), PyGetBackwardMap(), PyGetNodeImage(), PyGetNodePreImage(), PyGetNodeMap()



\begin{notice}{warning}{Warning:}
PyRunMethod() between the same two graph must be called before this function.
\end{notice}

\begin{notice}{note}{Note:}
This function creates datas so use it if necessary.
\end{notice}

\end{fulllineitems}

\index{PyGetBackwardMap() (in module PythonGedLib)}

\begin{fulllineitems}
\phantomsection\label{doc:PythonGedLib.PyGetBackwardMap}\pysiglinewithargsret{\sphinxcode{PythonGedLib.}\sphinxbfcode{PyGetBackwardMap}}{}{}
Returns the backward map (or the half of the adjacence matrix) between nodes of the two indicated graphs.
\begin{quote}\begin{description}
\item[{Parameters}] \leavevmode\begin{itemize}
\item {} 
\textbf{\texttt{g}} (\emph{\texttt{size\_t}}) -- The Id of the first compared graph

\item {} 
\textbf{\texttt{h}} (\emph{\texttt{size\_t}}) -- The Id of the second compared graph

\end{itemize}

\item[{Returns}] \leavevmode
The backward map to the adjacence matrix between nodes of the two graphs

\item[{Return type}] \leavevmode
list{[}npy\_uint32{]}

\end{description}\end{quote}


\sphinxstrong{See also:}


PyRunMethod(), PyGetUpperBound(), PyGetLowerBound(), PyGetForwardMap(), PyGetRuntime(), PyQuasimetricCost(), PyGetNodeMap(), PyGetAssignmentMatrix()



\begin{notice}{warning}{Warning:}
PyRunMethod() between the same two graph must be called before this function.
\end{notice}

\begin{notice}{note}{Note:}
I don't know how to connect the two map to reconstruct the adjacence matrix. Please come back when I know how it's work !
\end{notice}

\end{fulllineitems}

\index{PyGetDummyNode() (in module PythonGedLib)}

\begin{fulllineitems}
\phantomsection\label{doc:PythonGedLib.PyGetDummyNode}\pysiglinewithargsret{\sphinxcode{PythonGedLib.}\sphinxbfcode{PyGetDummyNode}}{}{}
Returns the ID of a dummy node.
\begin{quote}\begin{description}
\item[{Returns}] \leavevmode
The ID of the dummy node (18446744073709551614 for my computer, the hugest number possible)

\item[{Return type}] \leavevmode
size\_t

\end{description}\end{quote}

\begin{notice}{note}{Note:}
A dummy node is used when a node isn't associated to an other node.
\end{notice}

\end{fulllineitems}

\index{PyGetEditCostOptions() (in module PythonGedLib)}

\begin{fulllineitems}
\phantomsection\label{doc:PythonGedLib.PyGetEditCostOptions}\pysiglinewithargsret{\sphinxcode{PythonGedLib.}\sphinxbfcode{PyGetEditCostOptions}}{}{}
Searchs the differents edit cost functions and returns the result.
\begin{quote}\begin{description}
\item[{Returns}] \leavevmode
The list of edit cost functions

\item[{Return type}] \leavevmode
list{[}string{]}

\end{description}\end{quote}

\begin{notice}{warning}{Warning:}
This function is useless for an external use. Please use directly listOfEditCostOptions.
\end{notice}

\begin{notice}{note}{Note:}
Prefer the listOfEditCostOptions attribute of this module.
\end{notice}

\end{fulllineitems}

\index{PyGetForwardMap() (in module PythonGedLib)}

\begin{fulllineitems}
\phantomsection\label{doc:PythonGedLib.PyGetForwardMap}\pysiglinewithargsret{\sphinxcode{PythonGedLib.}\sphinxbfcode{PyGetForwardMap}}{}{}
Returns the forward map (or the half of the adjacence matrix) between nodes of the two indicated graphs.
\begin{quote}\begin{description}
\item[{Parameters}] \leavevmode\begin{itemize}
\item {} 
\textbf{\texttt{g}} (\emph{\texttt{size\_t}}) -- The Id of the first compared graph

\item {} 
\textbf{\texttt{h}} (\emph{\texttt{size\_t}}) -- The Id of the second compared graph

\end{itemize}

\item[{Returns}] \leavevmode
The forward map to the adjacence matrix between nodes of the two graphs

\item[{Return type}] \leavevmode
list{[}npy\_uint32{]}

\end{description}\end{quote}


\sphinxstrong{See also:}


PyRunMethod(), PyGetUpperBound(), PyGetLowerBound(), PyGetBackwardMap(), PyGetRuntime(), PyQuasimetricCost(), PyGetNodeMap(), PyGetAssignmentMatrix()



\begin{notice}{warning}{Warning:}
PyRunMethod() between the same two graph must be called before this function.
\end{notice}

\begin{notice}{note}{Note:}
I don't know how to connect the two map to reconstruct the adjacence matrix. Please come back when I know how it's work !
\end{notice}

\end{fulllineitems}

\index{PyGetGraphAdjacenceMatrix() (in module PythonGedLib)}

\begin{fulllineitems}
\phantomsection\label{doc:PythonGedLib.PyGetGraphAdjacenceMatrix}\pysiglinewithargsret{\sphinxcode{PythonGedLib.}\sphinxbfcode{PyGetGraphAdjacenceMatrix}}{}{}
Searchs and returns the adjacence list of a graph, selected by its ID.
\begin{quote}\begin{description}
\item[{Parameters}] \leavevmode
\textbf{\texttt{graphID}} (\emph{\texttt{size\_t}}) -- The ID of the wanted graph

\item[{Returns}] \leavevmode
The adjacence list of the selected graph

\item[{Return type}] \leavevmode
list{[}list{[}size\_t{]}{]}

\end{description}\end{quote}

\begin{notice}{note}{Note:}
These functions allow to collect all the graph's informations.
\end{notice}

\end{fulllineitems}

\index{PyGetGraphClass() (in module PythonGedLib)}

\begin{fulllineitems}
\phantomsection\label{doc:PythonGedLib.PyGetGraphClass}\pysiglinewithargsret{\sphinxcode{PythonGedLib.}\sphinxbfcode{PyGetGraphClass}}{}{}
Returns the class of a graph with its ID.
\begin{quote}\begin{description}
\item[{Parameters}] \leavevmode
\textbf{\texttt{id}} (\emph{\texttt{size\_t}}) -- The ID of the wanted graph

\item[{Returns}] \leavevmode
The class of the graph which correpond to the ID

\item[{Return type}] \leavevmode
\href{https://docs.python.org/3/library/string.html\#module-string}{string}

\end{description}\end{quote}


\sphinxstrong{See also:}


PyGetGraphClass()



\begin{notice}{note}{Note:}
An empty string can be a class.
\end{notice}

\end{fulllineitems}

\index{PyGetGraphEdges() (in module PythonGedLib)}

\begin{fulllineitems}
\phantomsection\label{doc:PythonGedLib.PyGetGraphEdges}\pysiglinewithargsret{\sphinxcode{PythonGedLib.}\sphinxbfcode{PyGetGraphEdges}}{}{}
Searchs and returns all the edges on a graph, selected by its ID.
\begin{quote}\begin{description}
\item[{Parameters}] \leavevmode
\textbf{\texttt{graphID}} (\emph{\texttt{size\_t}}) -- The ID of the wanted graph

\item[{Returns}] \leavevmode
The list of edges on the selected graph

\item[{Return type}] \leavevmode
dict\{tuple(size\_t,size\_t) : dict\{string : string\}\}

\end{description}\end{quote}

\begin{notice}{note}{Note:}
These functions allow to collect all the graph's informations.
\end{notice}

\end{fulllineitems}

\index{PyGetGraphIds() (in module PythonGedLib)}

\begin{fulllineitems}
\phantomsection\label{doc:PythonGedLib.PyGetGraphIds}\pysiglinewithargsret{\sphinxcode{PythonGedLib.}\sphinxbfcode{PyGetGraphIds}}{}{}
Searchs the first and last IDs of the loaded graphs in the environment.
\begin{quote}\begin{description}
\item[{Returns}] \leavevmode
The pair of the first and the last graphs Ids

\item[{Return type}] \leavevmode
tuple(size\_t, size\_t)

\end{description}\end{quote}

\begin{notice}{note}{Note:}
Prefer this function if you have huges structures with lots of graphs.
\end{notice}

\end{fulllineitems}

\index{PyGetGraphInternalId() (in module PythonGedLib)}

\begin{fulllineitems}
\phantomsection\label{doc:PythonGedLib.PyGetGraphInternalId}\pysiglinewithargsret{\sphinxcode{PythonGedLib.}\sphinxbfcode{PyGetGraphInternalId}}{}{}
Searchs and returns the internal Id of a graph, selected by its ID.
\begin{quote}\begin{description}
\item[{Parameters}] \leavevmode
\textbf{\texttt{graphID}} (\emph{\texttt{size\_t}}) -- The ID of the wanted graph

\item[{Returns}] \leavevmode
The internal ID of the selected graph

\item[{Return type}] \leavevmode
size\_t

\end{description}\end{quote}

\begin{notice}{note}{Note:}
These functions allow to collect all the graph's informations.
\end{notice}

\end{fulllineitems}

\index{PyGetGraphName() (in module PythonGedLib)}

\begin{fulllineitems}
\phantomsection\label{doc:PythonGedLib.PyGetGraphName}\pysiglinewithargsret{\sphinxcode{PythonGedLib.}\sphinxbfcode{PyGetGraphName}}{}{}
Returns the name of a graph with its ID.
\begin{quote}\begin{description}
\item[{Parameters}] \leavevmode
\textbf{\texttt{id}} (\emph{\texttt{size\_t}}) -- The ID of the wanted graph

\item[{Returns}] \leavevmode
The name of the graph which correpond to the ID

\item[{Return type}] \leavevmode
\href{https://docs.python.org/3/library/string.html\#module-string}{string}

\end{description}\end{quote}


\sphinxstrong{See also:}


PyGetGraphClass()



\begin{notice}{note}{Note:}
An empty string can be a name.
\end{notice}

\end{fulllineitems}

\index{PyGetGraphNodeLabels() (in module PythonGedLib)}

\begin{fulllineitems}
\phantomsection\label{doc:PythonGedLib.PyGetGraphNodeLabels}\pysiglinewithargsret{\sphinxcode{PythonGedLib.}\sphinxbfcode{PyGetGraphNodeLabels}}{}{}
Searchs and returns all the labels of nodes on a graph, selected by its ID.
\begin{quote}\begin{description}
\item[{Parameters}] \leavevmode
\textbf{\texttt{graphID}} (\emph{\texttt{size\_t}}) -- The ID of the wanted graph

\item[{Returns}] \leavevmode
The list of labels's nodes on the selected graph

\item[{Return type}] \leavevmode
list{[}dict\{string : string\}{]}

\end{description}\end{quote}

\begin{notice}{note}{Note:}
These functions allow to collect all the graph's informations.
\end{notice}

\end{fulllineitems}

\index{PyGetGraphNumEdges() (in module PythonGedLib)}

\begin{fulllineitems}
\phantomsection\label{doc:PythonGedLib.PyGetGraphNumEdges}\pysiglinewithargsret{\sphinxcode{PythonGedLib.}\sphinxbfcode{PyGetGraphNumEdges}}{}{}
Searchs and returns the number of edges on a graph, selected by its ID.
\begin{quote}\begin{description}
\item[{Parameters}] \leavevmode
\textbf{\texttt{graphID}} (\emph{\texttt{size\_t}}) -- The ID of the wanted graph

\item[{Returns}] \leavevmode
The number of edges on the selected graph

\item[{Return type}] \leavevmode
size\_t

\end{description}\end{quote}

\begin{notice}{note}{Note:}
These functions allow to collect all the graph's informations.
\end{notice}

\end{fulllineitems}

\index{PyGetGraphNumNodes() (in module PythonGedLib)}

\begin{fulllineitems}
\phantomsection\label{doc:PythonGedLib.PyGetGraphNumNodes}\pysiglinewithargsret{\sphinxcode{PythonGedLib.}\sphinxbfcode{PyGetGraphNumNodes}}{}{}
Searchs and returns the number of nodes on a graph, selected by its ID.
\begin{quote}\begin{description}
\item[{Parameters}] \leavevmode
\textbf{\texttt{graphID}} (\emph{\texttt{size\_t}}) -- The ID of the wanted graph

\item[{Returns}] \leavevmode
The number of nodes on the selected graph

\item[{Return type}] \leavevmode
size\_t

\end{description}\end{quote}

\begin{notice}{note}{Note:}
These functions allow to collect all the graph's informations.
\end{notice}

\end{fulllineitems}

\index{PyGetInitOptions() (in module PythonGedLib)}

\begin{fulllineitems}
\phantomsection\label{doc:PythonGedLib.PyGetInitOptions}\pysiglinewithargsret{\sphinxcode{PythonGedLib.}\sphinxbfcode{PyGetInitOptions}}{}{}
Searchs the differents initialization parameters for the environment computation for graphs and returns the result.
\begin{quote}\begin{description}
\item[{Returns}] \leavevmode
The list of options to initialize the computation environment

\item[{Return type}] \leavevmode
list{[}string{]}

\end{description}\end{quote}

\begin{notice}{warning}{Warning:}
This function is useless for an external use. Please use directly listOfInitOptions.
\end{notice}

\begin{notice}{note}{Note:}
Prefer the listOfInitOptions attribute of this module.
\end{notice}

\end{fulllineitems}

\index{PyGetInitime() (in module PythonGedLib)}

\begin{fulllineitems}
\phantomsection\label{doc:PythonGedLib.PyGetInitime}\pysiglinewithargsret{\sphinxcode{PythonGedLib.}\sphinxbfcode{PyGetInitime}}{}{}
Returns the initialization time.
\begin{quote}\begin{description}
\item[{Returns}] \leavevmode
The initialization time

\item[{Return type}] \leavevmode
double

\end{description}\end{quote}

\end{fulllineitems}

\index{PyGetLowerBound() (in module PythonGedLib)}

\begin{fulllineitems}
\phantomsection\label{doc:PythonGedLib.PyGetLowerBound}\pysiglinewithargsret{\sphinxcode{PythonGedLib.}\sphinxbfcode{PyGetLowerBound}}{}{}~\begin{quote}

Returns the lower bound of the edit distance cost between two graphs g and h.
\end{quote}
\begin{quote}\begin{description}
\item[{Parameters}] \leavevmode\begin{itemize}
\item {} 
\textbf{\texttt{g}} (\emph{\texttt{size\_t}}) -- The Id of the first compared graph

\item {} 
\textbf{\texttt{h}} (\emph{\texttt{size\_t}}) -- The Id of the second compared graph

\end{itemize}

\item[{Returns}] \leavevmode
The lower bound of the edit distance cost

\item[{Return type}] \leavevmode
double

\end{description}\end{quote}


\sphinxstrong{See also:}


PyRunMethod(), PyGetUpperBound(), PyGetForwardMap(), PyGetBackwardMap(), PyGetRuntime(), PyQuasimetricCost()



\begin{notice}{warning}{Warning:}
PyRunMethod() between the same two graph must be called before this function.
\end{notice}

\begin{notice}{note}{Note:}
This function can be ignored, because lower bound doesn't have a crucial utility.
\end{notice}

\end{fulllineitems}

\index{PyGetMethodOptions() (in module PythonGedLib)}

\begin{fulllineitems}
\phantomsection\label{doc:PythonGedLib.PyGetMethodOptions}\pysiglinewithargsret{\sphinxcode{PythonGedLib.}\sphinxbfcode{PyGetMethodOptions}}{}{}
Searchs the differents method for edit distance computation between graphs and returns the result.
\begin{quote}\begin{description}
\item[{Returns}] \leavevmode
The list of method to compute the edit distance between graphs

\item[{Return type}] \leavevmode
list{[}string{]}

\end{description}\end{quote}

\begin{notice}{warning}{Warning:}
This function is useless for an external use. Please use directly listOfMethodOptions.
\end{notice}

\begin{notice}{note}{Note:}
Prefer the listOfMethodOptions attribute of this module.
\end{notice}

\end{fulllineitems}

\index{PyGetNodeImage() (in module PythonGedLib)}

\begin{fulllineitems}
\phantomsection\label{doc:PythonGedLib.PyGetNodeImage}\pysiglinewithargsret{\sphinxcode{PythonGedLib.}\sphinxbfcode{PyGetNodeImage}}{}{}
Returns the node's image in the adjacence matrix, if it exists.
\begin{quote}\begin{description}
\item[{Parameters}] \leavevmode\begin{itemize}
\item {} 
\textbf{\texttt{g}} (\emph{\texttt{size\_t}}) -- The Id of the first compared graph

\item {} 
\textbf{\texttt{h}} (\emph{\texttt{size\_t}}) -- The Id of the second compared graph

\item {} 
\textbf{\texttt{nodeID}} -- The ID of the node which you want to see the image

\end{itemize}

\end{description}\end{quote}

:type nodeID size\_t
:return: The ID of the image node
:rtype: size\_t


\sphinxstrong{See also:}


PyRunMethod(), PyGetForwardMap(), PyGetBackwardMap(), PyGetNodePreImage(), PyGetNodeMap(), PyGetAssignmentMatrix()



\begin{notice}{warning}{Warning:}
PyRunMethod() between the same two graph must be called before this function.
\end{notice}

\begin{notice}{note}{Note:}
Use BackwardMap's Node to find its images ! You can also use PyGetForwardMap() and PyGetBackwardMap().
\end{notice}

\end{fulllineitems}

\index{PyGetNodeMap() (in module PythonGedLib)}

\begin{fulllineitems}
\phantomsection\label{doc:PythonGedLib.PyGetNodeMap}\pysiglinewithargsret{\sphinxcode{PythonGedLib.}\sphinxbfcode{PyGetNodeMap}}{}{}
Returns the Node Map, like C++ NodeMap.
\begin{quote}\begin{description}
\item[{Parameters}] \leavevmode\begin{itemize}
\item {} 
\textbf{\texttt{g}} (\emph{\texttt{size\_t}}) -- The Id of the first compared graph

\item {} 
\textbf{\texttt{h}} (\emph{\texttt{size\_t}}) -- The Id of the second compared graph

\end{itemize}

\item[{Returns}] \leavevmode
The Node Map between the two selected graph.

\item[{Return type}] \leavevmode
list{[}tuple(size\_t, size\_t){]}

\end{description}\end{quote}


\sphinxstrong{See also:}


PyRunMethod(), PyGetForwardMap(), PyGetBackwardMap(), PyGetNodeImage(), PyGetNodePreImage(), PyGetAssignmentMatrix()



\begin{notice}{warning}{Warning:}
PyRunMethod() between the same two graph must be called before this function.
\end{notice}

\begin{notice}{note}{Note:}
This function creates datas so use it if necessary, however you can understand how assignement works with this example.
\end{notice}

\end{fulllineitems}

\index{PyGetNodePreImage() (in module PythonGedLib)}

\begin{fulllineitems}
\phantomsection\label{doc:PythonGedLib.PyGetNodePreImage}\pysiglinewithargsret{\sphinxcode{PythonGedLib.}\sphinxbfcode{PyGetNodePreImage}}{}{}
Returns the node's preimage in the adjacence matrix, if it exists.
\begin{quote}\begin{description}
\item[{Parameters}] \leavevmode\begin{itemize}
\item {} 
\textbf{\texttt{g}} (\emph{\texttt{size\_t}}) -- The Id of the first compared graph

\item {} 
\textbf{\texttt{h}} (\emph{\texttt{size\_t}}) -- The Id of the second compared graph

\item {} 
\textbf{\texttt{nodeID}} -- The ID of the node which you want to see the preimage

\end{itemize}

\end{description}\end{quote}

:type nodeID size\_t
:return: The ID of the preimage node
:rtype: size\_t


\sphinxstrong{See also:}


PyRunMethod(), PyGetForwardMap(), PyGetBackwardMap(), PyGetNodeImage(), PyGetNodeMap(), PyGetAssignmentMatrix()



\begin{notice}{warning}{Warning:}
PyRunMethod() between the same two graph must be called before this function.
\end{notice}

\begin{notice}{note}{Note:}
Use ForwardMap's Node to find its images ! You can also use PyGetForwardMap() and PyGetBackwardMap().
\end{notice}

\end{fulllineitems}

\index{PyGetOriginalNodeIds() (in module PythonGedLib)}

\begin{fulllineitems}
\phantomsection\label{doc:PythonGedLib.PyGetOriginalNodeIds}\pysiglinewithargsret{\sphinxcode{PythonGedLib.}\sphinxbfcode{PyGetOriginalNodeIds}}{}{}
Searchs and returns all th Ids of nodes on a graph, selected by its ID.
\begin{quote}\begin{description}
\item[{Parameters}] \leavevmode
\textbf{\texttt{graphID}} (\emph{\texttt{size\_t}}) -- The ID of the wanted graph

\item[{Returns}] \leavevmode
The list of IDs's nodes on the selected graph

\item[{Return type}] \leavevmode
list{[}string{]}

\end{description}\end{quote}

\begin{notice}{note}{Note:}
These functions allow to collect all the graph's informations.
\end{notice}

\end{fulllineitems}

\index{PyGetRuntime() (in module PythonGedLib)}

\begin{fulllineitems}
\phantomsection\label{doc:PythonGedLib.PyGetRuntime}\pysiglinewithargsret{\sphinxcode{PythonGedLib.}\sphinxbfcode{PyGetRuntime}}{}{}
Returns the runtime to compute the edit distance cost between two graphs g and h
\begin{quote}\begin{description}
\item[{Parameters}] \leavevmode\begin{itemize}
\item {} 
\textbf{\texttt{g}} (\emph{\texttt{size\_t}}) -- The Id of the first compared graph

\item {} 
\textbf{\texttt{h}} (\emph{\texttt{size\_t}}) -- The Id of the second compared graph

\end{itemize}

\item[{Returns}] \leavevmode
The runtime of the computation of edit distance cost between the two selected graphs

\item[{Return type}] \leavevmode
double

\end{description}\end{quote}


\sphinxstrong{See also:}


PyRunMethod(), PyGetUpperBound(), PyGetLowerBound(),  PyGetForwardMap(), PyGetBackwardMap(), PyQuasimetricCost()



\begin{notice}{warning}{Warning:}
PyRunMethod() between the same two graph must be called before this function.
\end{notice}

\begin{notice}{note}{Note:}
Python is a bit longer than C++ due to the functions's encapsulate.
\end{notice}

\end{fulllineitems}

\index{PyGetUpperBound() (in module PythonGedLib)}

\begin{fulllineitems}
\phantomsection\label{doc:PythonGedLib.PyGetUpperBound}\pysiglinewithargsret{\sphinxcode{PythonGedLib.}\sphinxbfcode{PyGetUpperBound}}{}{}
Returns the upper bound of the edit distance cost between two graphs g and h.
\begin{quote}\begin{description}
\item[{Parameters}] \leavevmode\begin{itemize}
\item {} 
\textbf{\texttt{g}} (\emph{\texttt{size\_t}}) -- The Id of the first compared graph

\item {} 
\textbf{\texttt{h}} (\emph{\texttt{size\_t}}) -- The Id of the second compared graph

\end{itemize}

\item[{Returns}] \leavevmode
The upper bound of the edit distance cost

\item[{Return type}] \leavevmode
double

\end{description}\end{quote}


\sphinxstrong{See also:}


PyRunMethod(), PyGetLowerBound(), PyGetForwardMap(), PyGetBackwardMap(), PyGetRuntime(), PyQuasimetricCost()



\begin{notice}{warning}{Warning:}
PyRunMethod() between the same two graph must be called before this function.
\end{notice}

\begin{notice}{note}{Note:}
The upper bound is equivalent to the result of the pessimist edit distance cost. Methods are heuristics so the library can't compute the real perfect result because it's NP-Hard problem.
\end{notice}

\end{fulllineitems}

\index{PyHungarianLSAP() (in module PythonGedLib)}

\begin{fulllineitems}
\phantomsection\label{doc:PythonGedLib.PyHungarianLSAP}\pysiglinewithargsret{\sphinxcode{PythonGedLib.}\sphinxbfcode{PyHungarianLSAP}}{}{}
Applies the hungarian algorithm (LSAP) on a matrix Cost.
\begin{quote}\begin{description}
\item[{Parameters}] \leavevmode
\textbf{\texttt{matrixCost}} (\emph{\texttt{vector{[}vector{[}size\_t{]}{]}}}) -- The matrix Cost

\item[{Returns}] \leavevmode
The values of rho, varrho, u and v, in this order

\item[{Return type}] \leavevmode
vector{[}vector{[}size\_t{]}{]}

\end{description}\end{quote}

\end{fulllineitems}

\index{PyHungarianLSAPE() (in module PythonGedLib)}

\begin{fulllineitems}
\phantomsection\label{doc:PythonGedLib.PyHungarianLSAPE}\pysiglinewithargsret{\sphinxcode{PythonGedLib.}\sphinxbfcode{PyHungarianLSAPE}}{}{}
Applies the hungarian algorithm (LSAPE) on a matrix Cost.
\begin{quote}\begin{description}
\item[{Parameters}] \leavevmode
\textbf{\texttt{matrixCost}} (\emph{\texttt{vector{[}vector{[}double{]}{]}}}) -- The matrix Cost

\item[{Returns}] \leavevmode
The values of rho, varrho, u and v, in this order

\item[{Return type}] \leavevmode
vector{[}vector{[}double{]}{]}

\end{description}\end{quote}

\end{fulllineitems}

\index{PyInitEnv() (in module PythonGedLib)}

\begin{fulllineitems}
\phantomsection\label{doc:PythonGedLib.PyInitEnv}\pysiglinewithargsret{\sphinxcode{PythonGedLib.}\sphinxbfcode{PyInitEnv}}{}{}
Initializes the environment with the chosen edit cost function and graphs.
\begin{quote}\begin{description}
\item[{Parameters}] \leavevmode
\textbf{\texttt{initOption}} (\href{https://docs.python.org/3/library/string.html\#module-string}{\emph{\texttt{string}}}) -- The name of the init option, ``EAGER\_WITHOUT\_SHUFFLED\_COPIES'' by default

\end{description}\end{quote}


\sphinxstrong{See also:}


listOfInitOptions



\begin{notice}{warning}{Warning:}
No modification were allowed after initialization. Try to make sure your choices is correct. You can though clear or add a graph, but recall PyInitEnv() after that.
\end{notice}

\begin{notice}{note}{Note:}
Try to make sure the option exists with listOfInitOptions or choose no options, raise an error otherwise.
\end{notice}

\end{fulllineitems}

\index{PyInitMethod() (in module PythonGedLib)}

\begin{fulllineitems}
\phantomsection\label{doc:PythonGedLib.PyInitMethod}\pysiglinewithargsret{\sphinxcode{PythonGedLib.}\sphinxbfcode{PyInitMethod}}{}{}
Inits the environment with the set method.


\sphinxstrong{See also:}


PySetMethod(), listOfMethodOptions



\begin{notice}{note}{Note:}
Call this function after set the method. You can't launch computation or change the method after that.
\end{notice}

\end{fulllineitems}

\index{PyIsInitialized() (in module PythonGedLib)}

\begin{fulllineitems}
\phantomsection\label{doc:PythonGedLib.PyIsInitialized}\pysiglinewithargsret{\sphinxcode{PythonGedLib.}\sphinxbfcode{PyIsInitialized}}{}{}
Checks and returns if the computation environment is initialized or not.
\begin{quote}\begin{description}
\item[{Returns}] \leavevmode
True if it's initialized, False otherwise

\item[{Return type}] \leavevmode
\href{https://docs.python.org/3/library/functions.html\#bool}{bool}

\end{description}\end{quote}

\begin{notice}{note}{Note:}
This function exists for internals verifications but you can use it for your code.
\end{notice}

\end{fulllineitems}

\index{PyLoadGXLGraph() (in module PythonGedLib)}

\begin{fulllineitems}
\phantomsection\label{doc:PythonGedLib.PyLoadGXLGraph}\pysiglinewithargsret{\sphinxcode{PythonGedLib.}\sphinxbfcode{PyLoadGXLGraph}}{}{}
Loads some GXL graphes on the environment which is in a same folder, and present in the XMLfile.
\begin{quote}\begin{description}
\item[{Parameters}] \leavevmode\begin{itemize}
\item {} 
\textbf{\texttt{pathFolder}} (\href{https://docs.python.org/3/library/string.html\#module-string}{\emph{\texttt{string}}}) -- The folder's path which contains GXL graphs

\item {} 
\textbf{\texttt{pathXML}} (\href{https://docs.python.org/3/library/string.html\#module-string}{\emph{\texttt{string}}}) -- The XML's path which indicates which graphes you want to load

\end{itemize}

\end{description}\end{quote}

\begin{notice}{note}{Note:}
You can call this function multiple times if you want, but not after an init call.
\end{notice}

\end{fulllineitems}

\index{PyQuasimetricCost() (in module PythonGedLib)}

\begin{fulllineitems}
\phantomsection\label{doc:PythonGedLib.PyQuasimetricCost}\pysiglinewithargsret{\sphinxcode{PythonGedLib.}\sphinxbfcode{PyQuasimetricCost}}{}{}
Checks and returns if the edit costs are quasimetric.
\begin{quote}\begin{description}
\item[{Parameters}] \leavevmode\begin{itemize}
\item {} 
\textbf{\texttt{g}} (\emph{\texttt{size\_t}}) -- The Id of the first compared graph

\item {} 
\textbf{\texttt{h}} (\emph{\texttt{size\_t}}) -- The Id of the second compared graph

\end{itemize}

\item[{Returns}] \leavevmode
True if it's verified, False otherwise

\item[{Return type}] \leavevmode
\href{https://docs.python.org/3/library/functions.html\#bool}{bool}

\end{description}\end{quote}


\sphinxstrong{See also:}


PyRunMethod(), PyGetUpperBound(), PyGetLowerBound(),  PyGetForwardMap(), PyGetBackwardMap(), PyGetRuntime()



\begin{notice}{warning}{Warning:}
PyRunMethod() between the same two graph must be called before this function.
\end{notice}

\end{fulllineitems}

\index{PyRestartEnv() (in module PythonGedLib)}

\begin{fulllineitems}
\phantomsection\label{doc:PythonGedLib.PyRestartEnv}\pysiglinewithargsret{\sphinxcode{PythonGedLib.}\sphinxbfcode{PyRestartEnv}}{}{}
Restarts the environment variable. All data related to it will be delete.

\begin{notice}{warning}{Warning:}
This function deletes all graphs, computations and more so make sure you don't need anymore your environment.
\end{notice}

\begin{notice}{note}{Note:}
You can now delete and add somes graphs after initialization so you can avoid this function.
\end{notice}

\end{fulllineitems}

\index{PyRunMethod() (in module PythonGedLib)}

\begin{fulllineitems}
\phantomsection\label{doc:PythonGedLib.PyRunMethod}\pysiglinewithargsret{\sphinxcode{PythonGedLib.}\sphinxbfcode{PyRunMethod}}{}{}
Computes the edit distance between two graphs g and h, with the edit cost function and method computation selected.
\begin{quote}\begin{description}
\item[{Parameters}] \leavevmode\begin{itemize}
\item {} 
\textbf{\texttt{g}} (\emph{\texttt{size\_t}}) -- The Id of the first graph to compare

\item {} 
\textbf{\texttt{h}} (\emph{\texttt{size\_t}}) -- The Id of the second graph to compare

\end{itemize}

\end{description}\end{quote}


\sphinxstrong{See also:}


PyGetUpperBound(), PyGetLowerBound(),  PyGetForwardMap(), PyGetBackwardMap(), PyGetRuntime(), PyQuasimetricCost()



\begin{notice}{note}{Note:}
This function only compute the distance between two graphs, without returning a result. Use the differents function to see the result between the two graphs.
\end{notice}

\end{fulllineitems}

\index{PySetEditCost() (in module PythonGedLib)}

\begin{fulllineitems}
\phantomsection\label{doc:PythonGedLib.PySetEditCost}\pysiglinewithargsret{\sphinxcode{PythonGedLib.}\sphinxbfcode{PySetEditCost}}{}{}
Sets an edit cost function to the environment, if its exists.
\begin{quote}\begin{description}
\item[{Parameters}] \leavevmode\begin{itemize}
\item {} 
\textbf{\texttt{editCost}} (\href{https://docs.python.org/3/library/string.html\#module-string}{\emph{\texttt{string}}}) -- The name of the edit cost function

\item {} 
\textbf{\texttt{editCostConstant}} (\href{https://docs.python.org/3/library/stdtypes.html\#list}{\emph{\texttt{list}}}) -- The parameters you will add to the editCost, empty by default

\end{itemize}

\end{description}\end{quote}

\begin{notice}{note}{Note:}
Try to make sure the edit cost function exists with listOfEditCostOptions, raise an error otherwise.
\end{notice}

\end{fulllineitems}

\index{PySetMethod() (in module PythonGedLib)}

\begin{fulllineitems}
\phantomsection\label{doc:PythonGedLib.PySetMethod}\pysiglinewithargsret{\sphinxcode{PythonGedLib.}\sphinxbfcode{PySetMethod}}{}{}
Sets a computation method to the environment, if its exists.
\begin{quote}\begin{description}
\item[{Parameters}] \leavevmode\begin{itemize}
\item {} 
\textbf{\texttt{method}} (\href{https://docs.python.org/3/library/string.html\#module-string}{\emph{\texttt{string}}}) -- The name of the computation method

\item {} 
\textbf{\texttt{options}} (\href{https://docs.python.org/3/library/string.html\#module-string}{\emph{\texttt{string}}}) -- The options of the method (like bash options), an empty string by default

\end{itemize}

\end{description}\end{quote}


\sphinxstrong{See also:}


PyInitMethod(), listOfMethodOptions



\begin{notice}{note}{Note:}
Try to make sure the edit cost function exists with listOfMethodOptions, raise an error otherwise. Call PyInitMethod() after your set.
\end{notice}

\end{fulllineitems}

\index{PySetPersonalEditCost() (in module PythonGedLib)}

\begin{fulllineitems}
\phantomsection\label{doc:PythonGedLib.PySetPersonalEditCost}\pysiglinewithargsret{\sphinxcode{PythonGedLib.}\sphinxbfcode{PySetPersonalEditCost}}{}{}
Sets an personal edit cost function to the environment.
\begin{quote}\begin{description}
\item[{Parameters}] \leavevmode
\textbf{\texttt{editCostConstant}} (\href{https://docs.python.org/3/library/stdtypes.html\#list}{\emph{\texttt{list}}}) -- The parameters you will add to the editCost, empty by default

\end{description}\end{quote}

\end{fulllineitems}

\index{addNxGraph() (in module PythonGedLib)}

\begin{fulllineitems}
\phantomsection\label{doc:PythonGedLib.addNxGraph}\pysiglinewithargsret{\sphinxcode{PythonGedLib.}\sphinxbfcode{addNxGraph}}{}{}
Add a Graph (made by networkx) on the environment. Be careful to respect the same format as GXL graphs for labelling nodes and edges.
\begin{quote}\begin{description}
\item[{Parameters}] \leavevmode\begin{itemize}
\item {} 
\textbf{\texttt{g}} (\emph{\texttt{networkx.graph}}) -- The graph to add (networkx graph)

\item {} 
\textbf{\texttt{ignoreDuplicates}} (\href{https://docs.python.org/3/library/functions.html\#bool}{\emph{\texttt{bool}}}) -- If True, duplicate edges are ignored, otherwise it's raise an error if an existing edge is added. True by default

\end{itemize}

\item[{Returns}] \leavevmode
The ID of the newly added graphe

\item[{Return type}] \leavevmode
size\_t

\end{description}\end{quote}

\begin{notice}{note}{Note:}
The NX graph must respect the GXL structure. Please see how a GXL graph is construct.
\end{notice}

\end{fulllineitems}

\index{addRandomGraph() (in module PythonGedLib)}

\begin{fulllineitems}
\phantomsection\label{doc:PythonGedLib.addRandomGraph}\pysiglinewithargsret{\sphinxcode{PythonGedLib.}\sphinxbfcode{addRandomGraph}}{}{}
Add a Graph (not GXL) on the environment. Be careful to respect the same format as GXL graphs for labelling nodes and edges.
\begin{quote}\begin{description}
\item[{Parameters}] \leavevmode\begin{itemize}
\item {} 
\textbf{\texttt{name}} (\href{https://docs.python.org/3/library/string.html\#module-string}{\emph{\texttt{string}}}) -- The name of the graph to add, can be an empty string

\item {} 
\textbf{\texttt{classe}} (\href{https://docs.python.org/3/library/string.html\#module-string}{\emph{\texttt{string}}}) -- The classe of the graph to add, can be an empty string

\item {} 
\textbf{\texttt{listOfNodes}} (\emph{\texttt{list{[}tuple(size\_t, dict\{string : string\}){]}}}) -- The list of nodes to add

\item {} 
\textbf{\texttt{listOfEdges}} (\emph{\texttt{list{[}tuple(tuple(size\_t,size\_t), dict\{string : string\}){]}}}) -- The list of edges to add

\item {} 
\textbf{\texttt{ignoreDuplicates}} (\href{https://docs.python.org/3/library/functions.html\#bool}{\emph{\texttt{bool}}}) -- If True, duplicate edges are ignored, otherwise it's raise an error if an existing edge is added. True by default

\end{itemize}

\item[{Returns}] \leavevmode
The ID of the newly added graphe

\item[{Return type}] \leavevmode
size\_t

\end{description}\end{quote}

\begin{notice}{note}{Note:}
The graph must respect the GXL structure. Please see how a GXL graph is construct.
\end{notice}

\end{fulllineitems}

\index{appel() (in module PythonGedLib)}

\begin{fulllineitems}
\phantomsection\label{doc:PythonGedLib.appel}\pysiglinewithargsret{\sphinxcode{PythonGedLib.}\sphinxbfcode{appel}}{}{}
Calls an example only in C++. Nothing usefull, that's why you must ignore this function.

\end{fulllineitems}

\index{computeEditDistanceOnGXlGraphs() (in module PythonGedLib)}

\begin{fulllineitems}
\phantomsection\label{doc:PythonGedLib.computeEditDistanceOnGXlGraphs}\pysiglinewithargsret{\sphinxcode{PythonGedLib.}\sphinxbfcode{computeEditDistanceOnGXlGraphs}}{}{}
Computes all the edit distance between each GXL graphs on the folder and the XMl file.
\begin{quote}\begin{description}
\item[{Parameters}] \leavevmode\begin{itemize}
\item {} 
\textbf{\texttt{pathFolder}} (\href{https://docs.python.org/3/library/string.html\#module-string}{\emph{\texttt{string}}}) -- The folder's path which contains GXL graphs

\item {} 
\textbf{\texttt{pathXML}} (\href{https://docs.python.org/3/library/string.html\#module-string}{\emph{\texttt{string}}}) -- The XML's path which indicates which graphes you want to load

\item {} 
\textbf{\texttt{editCost}} (\href{https://docs.python.org/3/library/string.html\#module-string}{\emph{\texttt{string}}}) -- The name of the edit cost function

\item {} 
\textbf{\texttt{method}} (\href{https://docs.python.org/3/library/string.html\#module-string}{\emph{\texttt{string}}}) -- The name of the computation method

\item {} 
\textbf{\texttt{options}} (\href{https://docs.python.org/3/library/string.html\#module-string}{\emph{\texttt{string}}}) -- The options of the method (like bash options), an empty string by default

\item {} 
\textbf{\texttt{initOption}} (\href{https://docs.python.org/3/library/string.html\#module-string}{\emph{\texttt{string}}}) -- The name of the init option, ``EAGER\_WITHOUT\_SHUFFLED\_COPIES'' by default

\end{itemize}

\item[{Returns}] \leavevmode
The list of the first and last-1 ID of graphs

\item[{Return type}] \leavevmode
tuple(size\_t, size\_t)

\end{description}\end{quote}


\sphinxstrong{See also:}


listOfEditCostOptions, listOfMethodOptions, listOfInitOptions



\begin{notice}{note}{Note:}
Make sure each parameter exists with your architecture and these lists : listOfEditCostOptions, listOfMethodOptions, listOfInitOptions.
\end{notice}

\end{fulllineitems}

\index{computeEditDistanceOnNxGraphs() (in module PythonGedLib)}

\begin{fulllineitems}
\phantomsection\label{doc:PythonGedLib.computeEditDistanceOnNxGraphs}\pysiglinewithargsret{\sphinxcode{PythonGedLib.}\sphinxbfcode{computeEditDistanceOnNxGraphs}}{}{}
Computes all the edit distance between each NX graphs on the dataset.
\begin{quote}\begin{description}
\item[{Parameters}] \leavevmode\begin{itemize}
\item {} 
\textbf{\texttt{dataset}} (\emph{\texttt{list{[}networksx.graph{]}}}) -- The list of graphs to add and compute

\item {} 
\textbf{\texttt{classes}} (\href{https://docs.python.org/3/library/string.html\#module-string}{\emph{\texttt{string}}}) -- The classe of all the graph, can be an empty string

\item {} 
\textbf{\texttt{editCost}} (\href{https://docs.python.org/3/library/string.html\#module-string}{\emph{\texttt{string}}}) -- The name of the edit cost function

\item {} 
\textbf{\texttt{method}} (\href{https://docs.python.org/3/library/string.html\#module-string}{\emph{\texttt{string}}}) -- The name of the computation method

\item {} 
\textbf{\texttt{options}} (\href{https://docs.python.org/3/library/string.html\#module-string}{\emph{\texttt{string}}}) -- The options of the method (like bash options), an empty string by default

\item {} 
\textbf{\texttt{initOption}} (\href{https://docs.python.org/3/library/string.html\#module-string}{\emph{\texttt{string}}}) -- The name of the init option, ``EAGER\_WITHOUT\_SHUFFLED\_COPIES'' by default

\end{itemize}

\item[{Returns}] \leavevmode
Two matrix, the first with edit distances between graphs and the second the nodeMap between graphs. The result between g and h is one the {[}g{]}{[}h{]} coordinates.

\item[{Return type}] \leavevmode
list{[}list{[}double{]}{]}, list{[}list{[}list{[}tuple(size\_t, size\_t){]}{]}{]}

\end{description}\end{quote}


\sphinxstrong{See also:}


listOfEditCostOptions, listOfMethodOptions, listOfInitOptions



\begin{notice}{note}{Note:}
Make sure each parameter exists with your architecture and these lists : listOfEditCostOptions, listOfMethodOptions, listOfInitOptions. The structure of graphs must be similar as GXL.
\end{notice}

\end{fulllineitems}

\index{computeGedOnTwoGraphs() (in module PythonGedLib)}

\begin{fulllineitems}
\phantomsection\label{doc:PythonGedLib.computeGedOnTwoGraphs}\pysiglinewithargsret{\sphinxcode{PythonGedLib.}\sphinxbfcode{computeGedOnTwoGraphs}}{}{}
Computes the edit distance between two NX graphs.
\begin{quote}\begin{description}
\item[{Parameters}] \leavevmode\begin{itemize}
\item {} 
\textbf{\texttt{g1}} (\emph{\texttt{networksx.graph}}) -- The first graph to add and compute

\item {} 
\textbf{\texttt{g2}} (\emph{\texttt{networksx.graph}}) -- The second graph to add and compute

\item {} 
\textbf{\texttt{editCost}} (\href{https://docs.python.org/3/library/string.html\#module-string}{\emph{\texttt{string}}}) -- The name of the edit cost function

\item {} 
\textbf{\texttt{method}} (\href{https://docs.python.org/3/library/string.html\#module-string}{\emph{\texttt{string}}}) -- The name of the computation method

\item {} 
\textbf{\texttt{options}} (\href{https://docs.python.org/3/library/string.html\#module-string}{\emph{\texttt{string}}}) -- The options of the method (like bash options), an empty string by default

\item {} 
\textbf{\texttt{initOption}} (\href{https://docs.python.org/3/library/string.html\#module-string}{\emph{\texttt{string}}}) -- The name of the init option, ``EAGER\_WITHOUT\_SHUFFLED\_COPIES'' by default

\end{itemize}

\item[{Returns}] \leavevmode
The edit distance between the two graphs and the nodeMap between them.

\item[{Return type}] \leavevmode
double, list{[}tuple(size\_t, size\_t){]}

\end{description}\end{quote}


\sphinxstrong{See also:}


listOfEditCostOptions, listOfMethodOptions, listOfInitOptions



\begin{notice}{note}{Note:}
Make sure each parameter exists with your architecture and these lists : listOfEditCostOptions, listOfMethodOptions, listOfInitOptions. The structure of graphs must be similar as GXL.
\end{notice}

\end{fulllineitems}

\index{encodeYourMap() (in module PythonGedLib)}

\begin{fulllineitems}
\phantomsection\label{doc:PythonGedLib.encodeYourMap}\pysiglinewithargsret{\sphinxcode{PythonGedLib.}\sphinxbfcode{encodeYourMap}}{}{}
Encodes a string dictionnary to utf-8 for C++ functions
\begin{quote}\begin{description}
\item[{Parameters}] \leavevmode
\textbf{\texttt{map}} (\emph{\texttt{dict\{string : string\}}}) -- The map to encode

\item[{Returns}] \leavevmode
The encoded map

\item[{Return type}] \leavevmode
dict\{`b'string : `b'string\}

\end{description}\end{quote}

\begin{notice}{note}{Note:}
This function is used for type connection.
\end{notice}

\end{fulllineitems}



\chapter{How to add your own editCost class}
\label{editcost::doc}\label{editcost:how-to-add-your-own-editcost-class}
When you choose your cost function, you can decide some parameters to personalize the function. But if you have some graphs which its type doesn't correpond to the choices, you can create your edit cost function.

For this, you have to write it in C++.


\section{C++ side}
\label{editcost:c-side}
You class must inherit to EditCost class, which is an asbtract class. You can find it here : include/gedlib-master/src/edit\_costs

You can inspire you to the others to understand how to use it. You have to override these functions :
\begin{itemize}
\item {} 
virtual double node\_ins\_cost\_fun(const UserNodeLabel \& node\_label) const final;

\item {} 
virtual double node\_del\_cost\_fun(const UserNodeLabel \& node\_label) const final;

\item {} 
virtual double node\_rel\_cost\_fun(const UserNodeLabel \& node\_label\_1, const UserNodeLabel \& node\_label\_2) const final;

\item {} 
virtual double edge\_ins\_cost\_fun(const UserEdgeLabel \& edge\_label) const final;

\item {} 
virtual double edge\_del\_cost\_fun(const UserEdgeLabel \& edge\_label) const final;

\item {} 
virtual double edge\_rel\_cost\_fun(const UserEdgeLabel \& edge\_label\_1, const UserEdgeLabel \& edge\_label\_2) const final;

\end{itemize}

You can add some attributes for parameters use or more functions, but these are unavoidable.

When your class is ready, please go to the C++ Bind here : src/GedLibBind.cpp . The function is :
\begin{quote}

void setPersonalEditCost(std::vector\textless{}double\textgreater{} editCostConstants)\{env.set\_edit\_costs(Your EditCost Class(editCostConstants));\}
\end{quote}

You have just to initialize your class. Parameters aren't mandatory, empty by default. If your class doesn't have one, you can skip this. After that, you have to recompile the project.


\section{Python side}
\label{editcost:python-side}
For this, use setup.py with this command in a linux shell:

\begin{Verbatim}[commandchars=\\\{\}]
\PYG{n}{python3} \PYG{n}{setup}\PYG{o}{.}\PYG{n}{py} \PYG{n}{build\PYGZus{}ext} \PYG{o}{\PYGZhy{}}\PYG{o}{\PYGZhy{}}\PYG{n}{inplace}
\end{Verbatim}

You can also make it in Python 2.

Now you can use your edit cost function with the Python function PySetPersonalEditCost(editCostConstant).

If you want more informations on C++, you can check the documentation of the original library here : \url{https://github.com/dbblumenthal/gedlib}


\chapter{Examples}
\label{examples::doc}\label{examples:examples}

\section{Classique case with GXL graphs}
\label{examples:classique-case-with-gxl-graphs}
\begin{Verbatim}[commandchars=\\\{\}]
\PYG{n}{PythonGedLib}\PYG{o}{.}\PYG{n}{PyLoadGXLGraph}\PYG{p}{(}\PYG{l+s+s1}{\PYGZsq{}}\PYG{l+s+s1}{include/gedlib\PYGZhy{}master/data/datasets/Mutagenicity/data/}\PYG{l+s+s1}{\PYGZsq{}}\PYG{p}{,} \PYG{l+s+s1}{\PYGZsq{}}\PYG{l+s+s1}{collections/MUTA\PYGZus{}10.xml}\PYG{l+s+s1}{\PYGZsq{}}\PYG{p}{)}
\PYG{n}{listID} \PYG{o}{=} \PYG{n}{PythonGedLib}\PYG{o}{.}\PYG{n}{PyGetAllGraphIds}\PYG{p}{(}\PYG{p}{)}
\PYG{n}{PythonGedLib}\PYG{o}{.}\PYG{n}{PySetEditCost}\PYG{p}{(}\PYG{l+s+s2}{\PYGZdq{}}\PYG{l+s+s2}{CHEM\PYGZus{}1}\PYG{l+s+s2}{\PYGZdq{}}\PYG{p}{)}

\PYG{n}{PythonGedLib}\PYG{o}{.}\PYG{n}{PyInitEnv}\PYG{p}{(}\PYG{p}{)}

\PYG{n}{PythonGedLib}\PYG{o}{.}\PYG{n}{PySetMethod}\PYG{p}{(}\PYG{l+s+s2}{\PYGZdq{}}\PYG{l+s+s2}{IPFP}\PYG{l+s+s2}{\PYGZdq{}}\PYG{p}{,} \PYG{l+s+s2}{\PYGZdq{}}\PYG{l+s+s2}{\PYGZdq{}}\PYG{p}{)}
\PYG{n}{PythonGedLib}\PYG{o}{.}\PYG{n}{PyInitMethod}\PYG{p}{(}\PYG{p}{)}

\PYG{n}{g} \PYG{o}{=} \PYG{n}{listID}\PYG{p}{[}\PYG{l+m+mi}{0}\PYG{p}{]}
\PYG{n}{h} \PYG{o}{=} \PYG{n}{listID}\PYG{p}{[}\PYG{l+m+mi}{1}\PYG{p}{]}

\PYG{n}{PythonGedLib}\PYG{o}{.}\PYG{n}{PyRunMethod}\PYG{p}{(}\PYG{n}{g}\PYG{p}{,}\PYG{n}{h}\PYG{p}{)}

\PYG{k}{print}\PYG{p}{(}\PYG{l+s+s2}{\PYGZdq{}}\PYG{l+s+s2}{Node Map : }\PYG{l+s+s2}{\PYGZdq{}}\PYG{p}{,} \PYG{n}{PythonGedLib}\PYG{o}{.}\PYG{n}{PyGetNodeMap}\PYG{p}{(}\PYG{n}{g}\PYG{p}{,}\PYG{n}{h}\PYG{p}{)}\PYG{p}{)}
\PYG{k}{print} \PYG{p}{(}\PYG{l+s+s2}{\PYGZdq{}}\PYG{l+s+s2}{Upper Bound = }\PYG{l+s+s2}{\PYGZdq{}} \PYG{o}{+} \PYG{n+nb}{str}\PYG{p}{(}\PYG{n}{PythonGedLib}\PYG{o}{.}\PYG{n}{PyGetUpperBound}\PYG{p}{(}\PYG{n}{g}\PYG{p}{,}\PYG{n}{h}\PYG{p}{)}\PYG{p}{)} \PYG{o}{+} \PYG{l+s+s2}{\PYGZdq{}}\PYG{l+s+s2}{, Lower Bound = }\PYG{l+s+s2}{\PYGZdq{}} \PYG{o}{+} \PYG{n+nb}{str}\PYG{p}{(}\PYG{n}{PythonGedLib}\PYG{o}{.}\PYG{n}{PyGetLowerBound}\PYG{p}{(}\PYG{n}{g}\PYG{p}{,}\PYG{n}{h}\PYG{p}{)}\PYG{p}{)} \PYG{o}{+} \PYG{l+s+s2}{\PYGZdq{}}\PYG{l+s+s2}{, Runtime = }\PYG{l+s+s2}{\PYGZdq{}} \PYG{o}{+} \PYG{n+nb}{str}\PYG{p}{(}\PYG{n}{PythonGedLib}\PYG{o}{.}\PYG{n}{PyGetRuntime}\PYG{p}{(}\PYG{n}{g}\PYG{p}{,}\PYG{n}{h}\PYG{p}{)}\PYG{p}{)}\PYG{p}{)}
\end{Verbatim}

You can also use this function :

\begin{Verbatim}[commandchars=\\\{\}]
\PYG{n}{computeEditDistanceOnGXlGraphs}\PYG{p}{(}\PYG{n}{pathFolder}\PYG{p}{,} \PYG{n}{pathXML}\PYG{p}{,} \PYG{n}{editCost}\PYG{p}{,} \PYG{n}{method}\PYG{p}{,} \PYG{n}{options}\PYG{o}{=}\PYG{l+s+s2}{\PYGZdq{}}\PYG{l+s+s2}{\PYGZdq{}}\PYG{p}{,} \PYG{n}{initOption} \PYG{o}{=} \PYG{l+s+s2}{\PYGZdq{}}\PYG{l+s+s2}{EAGER\PYGZus{}WITHOUT\PYGZus{}SHUFFLED\PYGZus{}COPIES}\PYG{l+s+s2}{\PYGZdq{}}\PYG{p}{)}
\end{Verbatim}

This function compute all edit distance between all graphs, even itself. You can see the result with some functions and graphs IDs. Please see the documentation of the function for more informations.


\section{Classique case with NX graphs}
\label{examples:classique-case-with-nx-graphs}
\begin{Verbatim}[commandchars=\\\{\}]
\PYG{k}{for} \PYG{n}{graph} \PYG{o+ow}{in} \PYG{n}{dataset} \PYG{p}{:}
  \PYG{n}{PythonGedLib}\PYG{o}{.}\PYG{n}{addNxGraph}\PYG{p}{(}\PYG{n}{graph}\PYG{p}{,} \PYG{n}{classes}\PYG{p}{)}
\PYG{n}{listID} \PYG{o}{=} \PYG{n}{PythonGedLib}\PYG{o}{.}\PYG{n}{PyGetGraphIds}\PYG{p}{(}\PYG{p}{)}
\PYG{n}{PythonGedLib}\PYG{o}{.}\PYG{n}{PySetEditCost}\PYG{p}{(}\PYG{l+s+s2}{\PYGZdq{}}\PYG{l+s+s2}{CHEM\PYGZus{}1}\PYG{l+s+s2}{\PYGZdq{}}\PYG{p}{)}

\PYG{n}{PythonGedLib}\PYG{o}{.}\PYG{n}{PyInitEnv}\PYG{p}{(}\PYG{p}{)}

\PYG{n}{PythonGedLib}\PYG{o}{.}\PYG{n}{PySetMethod}\PYG{p}{(}\PYG{l+s+s2}{\PYGZdq{}}\PYG{l+s+s2}{IPFP}\PYG{l+s+s2}{\PYGZdq{}}\PYG{p}{,} \PYG{l+s+s2}{\PYGZdq{}}\PYG{l+s+s2}{\PYGZdq{}}\PYG{p}{)}
\PYG{n}{PythonGedLib}\PYG{o}{.}\PYG{n}{PyInitMethod}\PYG{p}{(}\PYG{p}{)}

\PYG{n}{g} \PYG{o}{=} \PYG{n}{listID}\PYG{p}{[}\PYG{l+m+mi}{0}\PYG{p}{]}
\PYG{n}{h} \PYG{o}{=} \PYG{n}{listID}\PYG{p}{[}\PYG{l+m+mi}{1}\PYG{p}{]}

\PYG{n}{PythonGedLib}\PYG{o}{.}\PYG{n}{PyRunMethod}\PYG{p}{(}\PYG{n}{g}\PYG{p}{,}\PYG{n}{h}\PYG{p}{)}

\PYG{k}{print}\PYG{p}{(}\PYG{l+s+s2}{\PYGZdq{}}\PYG{l+s+s2}{Node Map : }\PYG{l+s+s2}{\PYGZdq{}}\PYG{p}{,} \PYG{n}{PythonGedLib}\PYG{o}{.}\PYG{n}{PyGetNodeMap}\PYG{p}{(}\PYG{n}{g}\PYG{p}{,}\PYG{n}{h}\PYG{p}{)}\PYG{p}{)}
\PYG{k}{print} \PYG{p}{(}\PYG{l+s+s2}{\PYGZdq{}}\PYG{l+s+s2}{Upper Bound = }\PYG{l+s+s2}{\PYGZdq{}} \PYG{o}{+} \PYG{n+nb}{str}\PYG{p}{(}\PYG{n}{PythonGedLib}\PYG{o}{.}\PYG{n}{PyGetUpperBound}\PYG{p}{(}\PYG{n}{g}\PYG{p}{,}\PYG{n}{h}\PYG{p}{)}\PYG{p}{)} \PYG{o}{+} \PYG{l+s+s2}{\PYGZdq{}}\PYG{l+s+s2}{, Lower Bound = }\PYG{l+s+s2}{\PYGZdq{}} \PYG{o}{+} \PYG{n+nb}{str}\PYG{p}{(}\PYG{n}{PythonGedLib}\PYG{o}{.}\PYG{n}{PyGetLowerBound}\PYG{p}{(}\PYG{n}{g}\PYG{p}{,}\PYG{n}{h}\PYG{p}{)}\PYG{p}{)} \PYG{o}{+} \PYG{l+s+s2}{\PYGZdq{}}\PYG{l+s+s2}{, Runtime = }\PYG{l+s+s2}{\PYGZdq{}} \PYG{o}{+} \PYG{n+nb}{str}\PYG{p}{(}\PYG{n}{PythonGedLib}\PYG{o}{.}\PYG{n}{PyGetRuntime}\PYG{p}{(}\PYG{n}{g}\PYG{p}{,}\PYG{n}{h}\PYG{p}{)}\PYG{p}{)}\PYG{p}{)}
\end{Verbatim}

You can also use this function :

\begin{Verbatim}[commandchars=\\\{\}]
\PYG{n}{computeEditDistanceOnNxGraphs}\PYG{p}{(}\PYG{n}{dataset}\PYG{p}{,} \PYG{n}{classes}\PYG{p}{,} \PYG{n}{editCost}\PYG{p}{,} \PYG{n}{method}\PYG{p}{,} \PYG{n}{options}\PYG{p}{,} \PYG{n}{initOption} \PYG{o}{=} \PYG{l+s+s2}{\PYGZdq{}}\PYG{l+s+s2}{EAGER\PYGZus{}WITHOUT\PYGZus{}SHUFFLED\PYGZus{}COPIES}\PYG{l+s+s2}{\PYGZdq{}}\PYG{p}{)}
\end{Verbatim}

This function compute all edit distance between all graphs, even itself. You can see the result in the return and with some functions and graphs IDs. Please see the documentation of the function for more informations.

Or this function :

\begin{Verbatim}[commandchars=\\\{\}]
\PYG{n}{ccomputeGedOnTwoGraphs}\PYG{p}{(}\PYG{n}{g1}\PYG{p}{,}\PYG{n}{g2}\PYG{p}{,} \PYG{n}{editCost}\PYG{p}{,} \PYG{n}{method}\PYG{p}{,} \PYG{n}{options}\PYG{p}{,} \PYG{n}{initOption} \PYG{o}{=} \PYG{l+s+s2}{\PYGZdq{}}\PYG{l+s+s2}{EAGER\PYGZus{}WITHOUT\PYGZus{}SHUFFLED\PYGZus{}COPIES}\PYG{l+s+s2}{\PYGZdq{}}\PYG{p}{)}
\end{Verbatim}

This function allow to compute the edit distance just for two graphs. Please see the documentation of the function for more informations.


\section{Add a graph from scratch}
\label{examples:add-a-graph-from-scratch}
\begin{Verbatim}[commandchars=\\\{\}]
\PYG{n}{currentID} \PYG{o}{=} \PYG{n}{PythonGedLib}\PYG{o}{.}\PYG{n}{PyAddGraph}\PYG{p}{(}\PYG{p}{)}\PYG{p}{;}
\PYG{n}{PythonGedLib}\PYG{o}{.}\PYG{n}{PyAddNode}\PYG{p}{(}\PYG{n}{currentID}\PYG{p}{,} \PYG{l+s+s2}{\PYGZdq{}}\PYG{l+s+s2}{\PYGZus{}1}\PYG{l+s+s2}{\PYGZdq{}}\PYG{p}{,} \PYG{p}{\PYGZob{}}\PYG{l+s+s2}{\PYGZdq{}}\PYG{l+s+s2}{chem}\PYG{l+s+s2}{\PYGZdq{}} \PYG{p}{:} \PYG{l+s+s2}{\PYGZdq{}}\PYG{l+s+s2}{C}\PYG{l+s+s2}{\PYGZdq{}}\PYG{p}{\PYGZcb{}}\PYG{p}{)}
\PYG{n}{PythonGedLib}\PYG{o}{.}\PYG{n}{PyAddNode}\PYG{p}{(}\PYG{n}{currentID}\PYG{p}{,} \PYG{l+s+s2}{\PYGZdq{}}\PYG{l+s+s2}{\PYGZus{}2}\PYG{l+s+s2}{\PYGZdq{}}\PYG{p}{,} \PYG{p}{\PYGZob{}}\PYG{l+s+s2}{\PYGZdq{}}\PYG{l+s+s2}{chem}\PYG{l+s+s2}{\PYGZdq{}} \PYG{p}{:} \PYG{l+s+s2}{\PYGZdq{}}\PYG{l+s+s2}{O}\PYG{l+s+s2}{\PYGZdq{}}\PYG{p}{\PYGZcb{}}\PYG{p}{)}
\PYG{n}{PythonGedLib}\PYG{o}{.}\PYG{n}{PyAddEdge}\PYG{p}{(}\PYG{n}{currentID}\PYG{p}{,}\PYG{l+s+s2}{\PYGZdq{}}\PYG{l+s+s2}{\PYGZus{}1}\PYG{l+s+s2}{\PYGZdq{}}\PYG{p}{,} \PYG{l+s+s2}{\PYGZdq{}}\PYG{l+s+s2}{\PYGZus{}2}\PYG{l+s+s2}{\PYGZdq{}}\PYG{p}{,}  \PYG{p}{\PYGZob{}}\PYG{l+s+s2}{\PYGZdq{}}\PYG{l+s+s2}{valence}\PYG{l+s+s2}{\PYGZdq{}}\PYG{p}{:} \PYG{l+s+s2}{\PYGZdq{}}\PYG{l+s+s2}{1}\PYG{l+s+s2}{\PYGZdq{}}\PYG{p}{\PYGZcb{}} \PYG{p}{)}
\end{Verbatim}

Please make sure as the type are the same (string for Ids and a dictionnary for labels). If you want a symmetrical graph, you can use this function to ensure the symmetry :

\begin{Verbatim}[commandchars=\\\{\}]
\PYG{n}{PyAddSymmetricalEdge}\PYG{p}{(}\PYG{n}{graphID}\PYG{p}{,} \PYG{n}{tail}\PYG{p}{,} \PYG{n}{head}\PYG{p}{,} \PYG{n}{edgeLabel}\PYG{p}{)}
\end{Verbatim}

If you have a Nx structure, you can use directly this function :

\begin{Verbatim}[commandchars=\\\{\}]
\PYG{n}{addNxGraph}\PYG{p}{(}\PYG{n}{g}\PYG{p}{,} \PYG{n}{classe}\PYG{p}{,} \PYG{n}{ignoreDuplicates}\PYG{o}{=}\PYG{n+nb+bp}{True}\PYG{p}{)}
\end{Verbatim}

Even if you have another structure, you can use this function :

\begin{Verbatim}[commandchars=\\\{\}]
\PYG{n}{addRandomGraph}\PYG{p}{(}\PYG{n}{name}\PYG{p}{,} \PYG{n}{classe}\PYG{p}{,} \PYG{n}{listOfNodes}\PYG{p}{,} \PYG{n}{listOfEdges}\PYG{p}{,} \PYG{n}{ignoreDuplicates}\PYG{o}{=}\PYG{n+nb+bp}{True}\PYG{p}{)}
\end{Verbatim}

Please read the documentation before using and respect the types.


\section{Median computation}
\label{examples:median-computation}
Coming soon ...
Please ask Benoît Gauzere for this example !


\section{Hungarian algorithm}
\label{examples:hungarian-algorithm}

\subsection{LSAPE}
\label{examples:lsape}
\begin{Verbatim}[commandchars=\\\{\}]
\PYG{n}{result} \PYG{o}{=} \PYG{n}{PythonGedLib}\PYG{o}{.}\PYG{n}{PyHungarianLSAPE}\PYG{p}{(}\PYG{n}{matrixCost}\PYG{p}{)}
\PYG{k}{print}\PYG{p}{(}\PYG{l+s+s2}{\PYGZdq{}}\PYG{l+s+s2}{Rho = }\PYG{l+s+s2}{\PYGZdq{}}\PYG{p}{,} \PYG{n}{result}\PYG{p}{[}\PYG{l+m+mi}{0}\PYG{p}{]}\PYG{p}{,} \PYG{l+s+s2}{\PYGZdq{}}\PYG{l+s+s2}{ Varrho = }\PYG{l+s+s2}{\PYGZdq{}}\PYG{p}{,} \PYG{n}{result}\PYG{p}{[}\PYG{l+m+mi}{1}\PYG{p}{]}\PYG{p}{,} \PYG{l+s+s2}{\PYGZdq{}}\PYG{l+s+s2}{ u = }\PYG{l+s+s2}{\PYGZdq{}}\PYG{p}{,} \PYG{n}{result}\PYG{p}{[}\PYG{l+m+mi}{2}\PYG{p}{]}\PYG{p}{,} \PYG{l+s+s2}{\PYGZdq{}}\PYG{l+s+s2}{ v = }\PYG{l+s+s2}{\PYGZdq{}}\PYG{p}{,} \PYG{n}{result}\PYG{p}{[}\PYG{l+m+mi}{3}\PYG{p}{]}\PYG{p}{)}
\end{Verbatim}


\subsection{LSAP}
\label{examples:lsap}
\begin{Verbatim}[commandchars=\\\{\}]
\PYG{n}{result} \PYG{o}{=} \PYG{n}{PythonGedLib}\PYG{o}{.}\PYG{n}{PyHungarianLSAP}\PYG{p}{(}\PYG{n}{matrixCost}\PYG{p}{)}
\PYG{k}{print}\PYG{p}{(}\PYG{l+s+s2}{\PYGZdq{}}\PYG{l+s+s2}{Rho = }\PYG{l+s+s2}{\PYGZdq{}}\PYG{p}{,} \PYG{n}{result}\PYG{p}{[}\PYG{l+m+mi}{0}\PYG{p}{]}\PYG{p}{,} \PYG{l+s+s2}{\PYGZdq{}}\PYG{l+s+s2}{ u = }\PYG{l+s+s2}{\PYGZdq{}}\PYG{p}{,} \PYG{n}{result}\PYG{p}{[}\PYG{l+m+mi}{1}\PYG{p}{]}\PYG{p}{,} \PYG{l+s+s2}{\PYGZdq{}}\PYG{l+s+s2}{ v = }\PYG{l+s+s2}{\PYGZdq{}}\PYG{p}{,} \PYG{n}{result}\PYG{p}{[}\PYG{l+m+mi}{2}\PYG{p}{]}\PYG{p}{,} \PYG{l+s+s2}{\PYGZdq{}}\PYG{l+s+s2}{ Varrho = }\PYG{l+s+s2}{\PYGZdq{}}\PYG{p}{,} \PYG{n}{result}\PYG{p}{[}\PYG{l+m+mi}{3}\PYG{p}{]}\PYG{p}{)}
\end{Verbatim}


\chapter{Indices and tables}
\label{index:indices-and-tables}\begin{itemize}
\item {} 
\DUrole{xref,std,std-ref}{genindex}

\item {} 
\DUrole{xref,std,std-ref}{modindex}

\item {} 
\DUrole{xref,std,std-ref}{search}

\end{itemize}


\renewcommand{\indexname}{Python Module Index}
\begin{theindex}
\def\bigletter#1{{\Large\sffamily#1}\nopagebreak\vspace{1mm}}
\bigletter{p}
\item {\texttt{PythonGedLib}}, \pageref{doc:module-PythonGedLib}
\end{theindex}

\renewcommand{\indexname}{Index}
\printindex
\end{document}
